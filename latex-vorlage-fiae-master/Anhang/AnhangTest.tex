\begin{table}[!ht]
    \tabelleAnhang{Systeminformation}{tab:Systeminformation}{Systeminformation.tex}
    \caption{Hardwaredetails des Testsystems}
    \label{tab:tabletestsystem}
\end{table}

\subsection{Aufbau der Testumgebung}
\label{app:testaufbau}
Die zur Umsetzung dieses Kapitels benötigten Informationen entstammen unter anderem den hilfreichen Artikeln folgender Webseiten: \cite{networkdriverhacking}

\subsubsection{Implementierung der Virtuellen Maschinen}
Im Server-Manager fügen wir über \textit{Verwalten > Rollen und Features hinzufügen} den Hyper-V-Manager hinzu indem wir dem Assistenten folgen. Dieser gestattet es virtuelle Maschinen und Netzwerke zu installieren.
Als nächstes wird eine neue virtuelle Linux (Debian 7.1)  Maschine (Generation 1) aus einem Image erstellt. Dies geschieht mit Hilfe eines Assistenten. Sie bekommt einen virtuellen Prozessor und 1 \ac{GB} Arbeitsspeicher. Des weiteren wird bei der Installation eine 5 \ac{GB} große Festplatte für die Maschine erstellt und ihr zugewiesen. Als virtuellen Switch weisen wir ihr vorläufig den Netzwerkadapter des Hosts zu. Somit besitzt unsere Linux-\ac{VM} Internet. Um sie zu installieren, startet man nun die Maschine und verbindet sich zu ihr. Danach folgt man wie gewohnt den Installationsschritten wie bei einer physischen Maschine. Danach installieren wir den \ac{NTP}-Service. Ist die Grundkonfiguration fertig, wird die Maschine ausgeschaltet.
Die Installation der Windows 7 \ac{VM} erfolgt analog zu die der Linux \ac{VM}. Wir vergeben jedoch 4 \ac{GB} \ac{RAM} und erstellen eine mindestens 30 \ac{GB} große virtuelle Festplatte. Nach der Installation wird die Firewall wie in \nameref{sec:Implementierungsphase} eingerichtet. Zusätzlich werden noch nützliche Software wie putty oder winscp heruntergeladen.
Nach der Grundkonfiguration der beiden \ac{VM}s können diese nun dupliziert werden. Dazu muss man die virtuelle Maschine erst exportieren, um sie danach wieder zu importieren. Beim Import sollte man darauf achten, dass man \textit{eine neue eindeutige \ac{ID} } erstellt. Nachdem starten der importierten Maschine wird als erstes der Hostname geändert, um sie von der Originalen zu unterscheiden und um \ac{DNS}-Konflikte zu vermeiden.

\subsubsection{Implementierung des virtuellen Netzwerkes}
Virtuelle Netzwerke werden über das Hinzufügen virtueller Switche an den Netzwerkadaptern der virtuellen Maschine erstellt. Auf diesen lassen sich auch \ac{VLAN}s einrichten.
Die Installation eines solchen Switch wird ebenfalls vom Hyper-V-Manager mit einem Assistenten bereitgestellt. Für Testzwecke werden 2 \textit{private} Switche erstellt, da diese die direkte Kommunikation mit dem Host unterbinden und somit nicht die Router umgangen werden. Diese erhalten den Namen \ac{DMZ}- \bzw \ac{LAN}-Switch. Ein \textit{öffentlicher} Switch ist bereits vorhanden. Mit diesem ist der physische Netzwerkadapter des Hosts verbunden. Diese werden dann den \ac{VM}s entsprechend des \nameref{app:Netzplan}s zugeordnet. Für die Linux-\ac{VM}s, die als Router fungieren, muss \evtl noch ein zweiter Netzwerkadapter hinzugefügt werden. Nun können die Router und Clients (\text{Siehe Bild und Implementierung}) konfiguriert werden.

\subsubsection{Implementierung des \ac{DNS}-Servers}
Der \ac{DNS}-Server wird ebenfalls über den Server-Manager (unter \textit{Rollen und Features hinzufügen}) installiert. Diesen kann man nun über den \ac{DNS}-Manager verwalten. Es genügt eine \textit{Forward-Lookup}-Zone zu erstellen. Als Zonennamen wählen wir \textit{fritz.box} da bereits das Standard-Gateway darauf verweist. Dies ist die Domäne bzw. das \ac{DNS}-Suffix. Dieses Suffix wird auf den Windows-\ac{VM}s in den \ac{IP}v4-Einstellungen des Netzwerkadapters nachgetragen. Auf den Linux-\ac{VM}s tragen wir dies zusätzlich in die \verb+/etc/resolv.conf+ vor unserem \ac{DNS}-Server ein. \textbf{siehe resolv.conf oder selber schreiben]
Über den \ac{DNS}-Manager werden im Anschluss noch in der Zone \textit{fritz.box} unsere \ac{VM}s (A-Record) mit Namen und \ac{IP}-Adressen eingetragen. \textbf Siehe \ac{DNS}Manager.png}.

\subsubsection{Testen der Firewall}
Nachdem das Firewall-Script auf die Router kopiert und die \ac{DNS}-Server angepasst wurden, kann mit den Tests begonnen und die Firewall ggf.\ angepasst werden. Dazu speichern wir den Verlauf der erstellten Regeln als Log-Ausgabe in \verb+/var/log/firewall/firewallConfig+ ab. Die Ergebnisse unserer Tests finden sich als Übersicht in den folgenden Tabellen der Testprotokolle:


% Testprotokolle
\subsubsection{Testprotokolle}
\label{app:Testprotokolle}
% Description
\clearpage

% tables created with: http://www.tablesgenerator.com/latex_tables
% Please add the following required packages to your document preamble:
% \usepackage{graphicx}
% \usepackage[table,xcdraw]{xcolor}
% If you use beamer only pass "xcolor=table" option, i.e. \documentclass[xcolor=table]{beamer}
\begin{table}[!ht]
    \resizebox{\textwidth}{!}{%
        \begin{tabular}{llllll}
            \rowcolor[HTML]{009BA7} 
            \textbf{Service} & \textbf{Command} & \textbf{Source-IP} & \textbf{Destination-IP} & \textbf{Soll} & \textbf{Ist} \\
            ICMP & ping & 10.0.9.2 & 10.0.9.1 & ja & ja \\
            \rowcolor[HTML]{EFEFEF} 
            ICMP & ping & 10.0.9.3 & 10.0.9.1 & ja & ja \\
            ICMP & ping & 10.0.9.2 & 172.16.9.1 & ja & ja \\
            \rowcolor[HTML]{EFEFEF} 
            ICMP & ping & 10.0.9.3 & 172.16.9.1 & ja & ja \\
            ICMP & ping & 172.16.9.3 & 172.16.9.1 & ja & ja \\
            \rowcolor[HTML]{EFEFEF} 
            ICMP & ping & 172.16.9.3 & 172.16.9.2 & ja & ja \\
            ICMP & ping & 10.0.9.3 & 192.168.200.10 & ja & ja \\
            \rowcolor[HTML]{EFEFEF} 
            ICMP & ping & 192.168.200.10 & 10.0.9.3 & ja & ja \\
            ICMP & ping & 192.168.200.10 & 172.16.9.3 & ja & ja \\
            \rowcolor[HTML]{EFEFEF} 
            ICMP & ping & 192.168.200.10 & 192.168.200.109 & ja & ja \\
            ICMP & ping & 10.0.9.3 & 8.8.8.8 & ja & ja \\
            \rowcolor[HTML]{EFEFEF} 
            HTTP & http://172.16.9.3 & 192.168.200.10 & 192.168.200.109 & ja & ja \\
            HTTP & http://172.16.9.3 & 192.168.200.10 & 172.16.9.3 & ja & ja \\
            \rowcolor[HTML]{EFEFEF} 
            HTTP & http://172.16.9.3 & 10.0.9.3 & 192.168.200.109 & ja & ja \\
            HTTP & http://172.16.9.3 & 10.0.9.3 & 172.16.9.3 & ja & ja \\
            \rowcolor[HTML]{EFEFEF} 
            NTP & w32tm /stripchart /computer:192.168.200.1 & 10.0.9.2 & 192.168.200.1 & ja & ja \\
            NTP & w32tm /stripchart /computer:192.168.200.1 & 172.16.9.3 & 192.168.200.1 & ja & ja \\
            \rowcolor[HTML]{EFEFEF} 
            NTP & ntpq -p & 192.168.200.109 & 192.168.200.1 & ja & ja \\
            RDP & mstsc.exe & 10.0.9.2 & 172.16.9.3 & ja & ja \\
            \rowcolor[HTML]{EFEFEF} 
            RDP & mstsc.exe & 10.0.9.3 & 172.16.9.3 & ja & ja \\
            SSH & putty & 10.0.9.2 & 10.0.9.1 & ja & ja \\
            \rowcolor[HTML]{EFEFEF} 
            SSH & putty & 10.0.9.2 & 172.16.9.1 & ja & ja \\
            SSH & putty & 10.0.9.3 & 10.0.9.1 & ja & ja \\
            \rowcolor[HTML]{EFEFEF} 
            SSH & putty & 10.0.9.3 & 172.16.9.1 & ja & ja \\
            SSH & putty & 172.16.9.3 & 172.16.9.1 & ja & ja \\
            \rowcolor[HTML]{EFEFEF} 
            SSH & putty & 172.16.9.3 & 172.16.9.2 & ja & ja \\
            DNS & nslookup 172.16.9.1 & 10.0.9.3 & 192.168.200.10 & ja & ja \\
            \rowcolor[HTML]{EFEFEF} 
            DNS & nslookup Inside-Router & 172.16.9.3 & 192.168.200.10 & ja & ja \\
            DNS & nslookup 10.0.9.1 & 172.16.9.2 & 192.168.200.10 & ja & ja \\
            \rowcolor[HTML]{EFEFEF} 
            DNS & nslookup Client-PC & 192.168.200.109 & 192.168.200.10 & ja & ja
        \end{tabular}%
    }
    \caption{Aus - Aus}
    \label{tab:ausaus}
\end{table}

% Please add the following required packages to your document preamble:
% \usepackage{graphicx}
% \usepackage[table,xcdraw]{xcolor}
% If you use beamer only pass "xcolor=table" option, i.e. \documentclass[xcolor=table]{beamer}
\begin{table}[!ht]
    \resizebox{\textwidth}{!}{%
        \begin{tabular}{llllll}
            \rowcolor[HTML]{009BA7} 
            \textbf{Service} & \textbf{Command} & \textbf{Source-IP} & \textbf{Destination-IP} & \textbf{Soll} & \textbf{Ist} \\
            ICMP & ping & 10.0.9.2 & 10.0.9.1 & ja & ja \\
            \rowcolor[HTML]{EFEFEF} 
            ICMP & ping & 10.0.9.3 & 10.0.9.1 & nein & nein \\
            ICMP & ping & 10.0.9.2 & 172.16.9.1 & ja & ja \\
            \rowcolor[HTML]{EFEFEF} 
            ICMP & ping & 10.0.9.3 & 172.16.9.1 & nein & nein \\
            ICMP & ping & 172.16.9.3 & 172.16.9.1 & ja & ja \\
            \rowcolor[HTML]{EFEFEF} 
            ICMP & ping & 172.16.9.3 & 172.16.9.2 & nein & nein \\
            ICMP & ping & 10.0.9.3 & 192.168.200.10 & ja & ja \\
            \rowcolor[HTML]{EFEFEF} 
            ICMP & ping & 192.168.200.10 & 10.0.9.3 & nein & nein \\
            ICMP & ping & 192.168.200.10 & 172.16.9.3 & ja & ja \\
            \rowcolor[HTML]{EFEFEF} 
            ICMP & ping & 192.168.200.10 & 192.168.200.109 & ja & ja \\
            ICMP & ping & 10.0.9.3 & 8.8.8.8 & ja & ja \\
            \rowcolor[HTML]{EFEFEF} 
            HTTP & http://172.16.9.3 & 192.168.200.10 & 192.168.200.109 & ja & ja \\
            HTTP & http://172.16.9.3 & 192.168.200.10 & 172.16.9.3 & ja & ja \\
            \rowcolor[HTML]{EFEFEF} 
            HTTP & http://172.16.9.3 & 10.0.9.3 & 192.168.200.109 & nein & nein \\
            HTTP & http://172.16.9.3 & 10.0.9.3 & 172.16.9.3 & ja & ja \\
            \rowcolor[HTML]{EFEFEF} 
            NTP & w32tm /stripchart /computer:192.168.200.1 & 10.0.9.2 & 192.168.200.1 & ja & ja \\
            NTP & w32tm /stripchart /computer:192.168.200.1 & 172.16.9.3 & 192.168.200.1 & ja & ja \\
            \rowcolor[HTML]{EFEFEF} 
            NTP & ntpq -p & 192.168.200.109 & 192.168.200.1 & ja & ja \\
            RDP & mstsc.exe & 10.0.9.2 & 172.16.9.3 & ja & ja \\
            \rowcolor[HTML]{EFEFEF} 
            RDP & mstsc.exe & 10.0.9.3 & 172.16.9.3 & nein & nein \\
            SSH & putty & 10.0.9.2 & 10.0.9.1 & ja & ja \\
            \rowcolor[HTML]{EFEFEF} 
            SSH & putty & 10.0.9.2 & 172.16.9.1 & ja & ja \\
            SSH & putty & 10.0.9.3 & 10.0.9.1 & nein & nein \\
            \rowcolor[HTML]{EFEFEF} 
            SSH & putty & 10.0.9.3 & 172.16.9.1 & ja & ja \\
            SSH & putty & 172.16.9.3 & 172.16.9.1 & ja & ja \\
            \rowcolor[HTML]{EFEFEF} 
            SSH & putty & 172.16.9.3 & 172.16.9.2 & nein & nein \\
            DNS & nslookup 172.16.9.1 & 10.0.9.3 & 192.168.200.10 & ja & ja \\
            \rowcolor[HTML]{EFEFEF} 
            DNS & nslookup Inside-Router & 172.16.9.3 & 192.168.200.10 & ja & ja \\
            DNS & nslookup 10.0.9.1 & 172.16.9.2 & 192.168.200.10 & ja & ja \\
            \rowcolor[HTML]{EFEFEF} 
            DNS & nslookup Client-PC & 192.168.200.109 & 192.168.200.10 & ja & ja
        \end{tabular}%
    }
    \caption{Aus - An}
    \label{tab:ausan}
\end{table}

% Please add the following required packages to your document preamble:
% \usepackage{graphicx}
% \usepackage[table,xcdraw]{xcolor}
% If you use beamer only pass "xcolor=table" option, i.e. \documentclass[xcolor=table]{beamer}
\begin{table}[!ht]
    \resizebox{\textwidth}{!}{%
        \begin{tabular}{llllll}
            \rowcolor[HTML]{009BA7} 
            \textbf{Service} & \textbf{Command} & \textbf{Source-IP} & \textbf{Destination-IP} & \textbf{Soll} & \textbf{Ist} \\
            ICMP & ping & 10.0.9.2 & 10.0.9.1 & ja & ja \\
            \rowcolor[HTML]{EFEFEF} 
            ICMP & ping & 10.0.9.3 & 10.0.9.1 & ja & ja \\
            ICMP & ping & 10.0.9.2 & 172.16.9.1 & ja & ja \\
            \rowcolor[HTML]{EFEFEF} 
            ICMP & ping & 10.0.9.3 & 172.16.9.1 & nein & nein \\
            ICMP & ping & 172.16.9.3 & 172.16.9.1 & nein & nein \\
            \rowcolor[HTML]{EFEFEF} 
            ICMP & ping & 172.16.9.3 & 172.16.9.2 & ja & ja \\
            ICMP & ping & 10.0.9.3 & 192.168.200.10 & ja & ja \\
            \rowcolor[HTML]{EFEFEF} 
            ICMP & ping & 192.168.200.10 & 10.0.9.3 & nein & nein \\
            ICMP & ping & 192.168.200.10 & 172.16.9.3 & nein & nein \\
            \rowcolor[HTML]{EFEFEF} 
            ICMP & ping & 192.168.200.10 & 192.168.200.109 & nein & nein \\
            ICMP & ping & 10.0.9.3 & 8.8.8.8 & ja & ja \\
            \rowcolor[HTML]{EFEFEF} 
            HTTP & http://172.16.9.3 & 192.168.200.10 & 192.168.200.109 & ja & ja \\
            HTTP & http://172.16.9.3 & 192.168.200.10 & 172.16.9.3 & ja & ja \\
            \rowcolor[HTML]{EFEFEF} 
            HTTP & http://172.16.9.3 & 10.0.9.3 & 192.168.200.109 & nein & nein \\
            HTTP & http://172.16.9.3 & 10.0.9.3 & 172.16.9.3 & ja & ja \\
            \rowcolor[HTML]{EFEFEF} 
            NTP & w32tm /stripchart /computer:192.168.200.1 & 10.0.9.2 & 192.168.200.1 & ja & ja \\
            NTP & w32tm /stripchart /computer:192.168.200.1 & 172.16.9.3 & 192.168.200.1 & ja & ja \\
            \rowcolor[HTML]{EFEFEF} 
            NTP & ntpq -p & 192.168.200.109 & 192.168.200.1 & ja & ja \\
            RDP & mstsc.exe & 10.0.9.2 & 172.16.9.3 & ja & ja \\
            \rowcolor[HTML]{EFEFEF} 
            RDP & mstsc.exe & 10.0.9.3 & 172.16.9.3 & nein & nein \\
            SSH & putty & 10.0.9.2 & 10.0.9.1 & ja & ja \\
            \rowcolor[HTML]{EFEFEF} 
            SSH & putty & 10.0.9.2 & 172.16.9.1 & ja & ja \\
            SSH & putty & 10.0.9.3 & 10.0.9.1 & ja & ja \\
            \rowcolor[HTML]{EFEFEF} 
            SSH & putty & 10.0.9.3 & 172.16.9.1 & ja & ja \\
            SSH & putty & 172.16.9.3 & 172.16.9.1 & nein & nein \\
            \rowcolor[HTML]{EFEFEF} 
            SSH & putty & 172.16.9.3 & 172.16.9.2 & ja & ja \\
            DNS & nslookup 172.16.9.1 & 10.0.9.3 & 192.168.200.10 & ja & ja \\
            \rowcolor[HTML]{EFEFEF} 
            DNS & nslookup Inside-Router & 172.16.9.3 & 192.168.200.10 & ja & ja \\
            DNS & nslookup 10.0.9.1 & 172.16.9.2 & 192.168.200.10 & ja & ja \\
            \rowcolor[HTML]{EFEFEF} 
            DNS & nslookup Client-PC & 192.168.200.109 & 192.168.200.10 & ja & ja
        \end{tabular}%
    }
    \caption{An - Aus}
    \label{tab:anaus}
\end{table}

% Please add the following required packages to your document preamble:
% \usepackage{graphicx}
% \usepackage[table,xcdraw]{xcolor}
% If you use beamer only pass "xcolor=table" option, i.e. \documentclass[xcolor=table]{beamer}
\begin{table}[!ht]
    \resizebox{\textwidth}{!}{%
        \begin{tabular}{llllll}
            \rowcolor[HTML]{009BA7} 
            \textbf{Service} & \textbf{Command} & \textbf{Source-IP} & \textbf{Destination-IP} & \textbf{Soll} & \textbf{Ist} \\
            ICMP & ping & 10.0.9.2 & 10.0.9.1 & ja & ja \\
            \rowcolor[HTML]{EFEFEF} 
            ICMP & ping & 10.0.9.3 & 10.0.9.1 & ja & ja \\
            ICMP & ping & 10.0.9.2 & 172.16.9.1 & ja & ja \\
            \rowcolor[HTML]{EFEFEF} 
            ICMP & ping & 10.0.9.3 & 172.16.9.1 & nein & nein \\
            ICMP & ping & 172.16.9.3 & 172.16.9.1 & nein & nein \\
            \rowcolor[HTML]{EFEFEF} 
            ICMP & ping & 172.16.9.3 & 172.16.9.2 & ja & ja \\
            ICMP & ping & 10.0.9.3 & 192.168.200.10 & ja & ja \\
            \rowcolor[HTML]{EFEFEF} 
            ICMP & ping & 192.168.200.10 & 10.0.9.3 & nein & nein \\
            ICMP & ping & 192.168.200.10 & 172.16.9.3 & nein & nein \\
            \rowcolor[HTML]{EFEFEF} 
            ICMP & ping & 192.168.200.10 & 192.168.200.109 & nein & nein \\
            ICMP & ping & 10.0.9.3 & 8.8.8.8 & ja & ja \\
            \rowcolor[HTML]{EFEFEF} 
            HTTP & http://172.16.9.3 & 192.168.200.10 & 192.168.200.109 & ja & ja \\
            HTTP & http://172.16.9.3 & 192.168.200.10 & 172.16.9.3 & ja & ja \\
            \rowcolor[HTML]{EFEFEF} 
            HTTP & http://172.16.9.3 & 10.0.9.3 & 192.168.200.109 & nein & nein \\
            HTTP & http://172.16.9.3 & 10.0.9.3 & 172.16.9.3 & ja & ja \\
            \rowcolor[HTML]{EFEFEF} 
            NTP & w32tm /stripchart /computer:192.168.200.1 & 10.0.9.2 & 192.168.200.1 & ja & ja \\
            NTP & w32tm /stripchart /computer:192.168.200.1 & 172.16.9.3 & 192.168.200.1 & ja & ja \\
            \rowcolor[HTML]{EFEFEF} 
            NTP & ntpq -p & 192.168.200.109 & 192.168.200.1 & ja & ja \\
            RDP & mstsc.exe & 10.0.9.2 & 172.16.9.3 & ja & ja \\
            \rowcolor[HTML]{EFEFEF} 
            RDP & mstsc.exe & 10.0.9.3 & 172.16.9.3 & nein & nein \\
            SSH & putty & 10.0.9.2 & 10.0.9.1 & ja & ja \\
            \rowcolor[HTML]{EFEFEF} 
            SSH & putty & 10.0.9.2 & 172.16.9.1 & ja & ja \\
            SSH & putty & 10.0.9.3 & 10.0.9.1 & ja & ja \\
            \rowcolor[HTML]{EFEFEF} 
            SSH & putty & 10.0.9.3 & 172.16.9.1 & ja & ja \\
            SSH & putty & 172.16.9.3 & 172.16.9.1 & nein & nein \\
            \rowcolor[HTML]{EFEFEF} 
            SSH & putty & 172.16.9.3 & 172.16.9.2 & ja & ja \\
            DNS & nslookup 172.16.9.1 & 10.0.9.3 & 192.168.200.10 & ja & ja \\
            \rowcolor[HTML]{EFEFEF} 
            DNS & nslookup Inside-Router & 172.16.9.3 & 192.168.200.10 & ja & ja \\
            DNS & nslookup 10.0.9.1 & 172.16.9.2 & 192.168.200.10 & ja & ja \\
            \rowcolor[HTML]{EFEFEF} 
            DNS & nslookup Client-PC & 192.168.200.109 & 192.168.200.10 & ja & ja
        \end{tabular}%
    }
    \caption{An - An}
    \label{tab:anan}
\end{table}
\clearpage

\subsection{Firewall-Skripte}
\label{app:Firewall}

\subsubsection{firewall.sh (auf dem Outside-Router)}
\label{app:Firewall-Outside}
\lstinputlisting[language=sh]{Listings/outside/firewall.sh}

\subsubsection{firewall.sh (auf dem Inside-Router)}
\label{app:Firewall-Inside}
\lstinputlisting[language=sh]{Listings/inside/firewall.sh}
