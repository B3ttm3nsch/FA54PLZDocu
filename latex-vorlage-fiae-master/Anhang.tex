% !TEX root = Projektdokumentation.tex

\includepdf[scale=0.8,clip,trim=0cm 0cm 0cm 0cm,offset=0 -2cm,pages={1},pagecommand={\section{Anhang}\subsection{Schritt-für-Schritt Anleitung}\label{app:Anleitung}}]{AufbauEinerDMZ.pdf} \includepdf[scale=0.85,clip,trim=0cm 0cm 0cm 0cm,offset=0 -0.5cm,pages={2-5},pagecommand={}]{AufbauEinerDMZ.pdf}

\subsection{Detaillierte Zeitplanung}
\label{app:Zeitplanung}

\tabelleAnhang{ZeitplanungKomplett}

\input{Anhang/AnhangLastenheft.tex}
\clearpage

\subsection{Use Case-Diagramm}
\label{app:UseCase}
Use Case-Diagramme und weitere \acs{UML}-Diagramme kann man auch direkt mit \LaTeX{} zeichnen, siehe \zB \url{http://metauml.sourceforge.net/old/usecase-diagram.html}.
\begin{figure}[htb]
\centering
\includegraphicsKeepAspectRatio{UseCase.pdf}{0.7}
\caption{Use Case-Diagramm}
\end{figure}

\input{Anhang/AnhangPflichtenheft.tex}

\subsection{Netzplan}
\label{app:Netzplan}
Der Netzplan unserer \acs{DMZ}
\begin{figure}[htb]
\centering
\includegraphicsKeepAspectRatio{PLZNetzplanProjektumgebung.png}{1}
\caption{Netzplan der DMZ (Arbeitsgruppe 9)}
\end{figure}
\clearpage

\input{Anhang/AnhangEntwuerfe.tex}
\clearpage
\input{Anhang/AnhangScreenshots.tex}
\input{Anhang/AnhangDoc.tex}
\clearpage
\subsection{Beschreibung des Testaufbaues}
\label{app:Test}
\subsection{title}
\subsubsection{firewall.sh (auf dem Outside-Router)}
\label{app:Firewall-Outside}
\lstinputlisting[language=sh]{Listings/outside/firewall.sh}

\subsubsection{firewall.sh (auf dem Inside-Router)}
\label{app:Firewall-Inside}
\lstinputlisting[language=sh]{Listings/inside/firewall.sh}


\subsection{Klasse: ComparedNaturalModuleInformation}
\label{app:CNMI}
Kommentare und simple Getter/Setter werden nicht angezeigt.
\lstinputlisting[language=php]{Listings/cnmi.php}
\clearpage

\subsection{Klassendiagramm}
\label{app:Klassendiagramm}
Klassendiagramme und weitere \acs{UML}-Diagramme kann man auch direkt mit \LaTeX{} zeichnen, siehe \zB \url{http://metauml.sourceforge.net/old/class-diagram.html}.
\begin{figure}[htb]
\centering
\includegraphicsKeepAspectRatio{Klassendiagramm.pdf}{1}
\caption{Klassendiagramm}
\end{figure}
\clearpage

\input{Anhang/AnhangBenutzerDoku.tex}
