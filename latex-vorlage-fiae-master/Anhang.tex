% !TEX root = Projektdokumentation.tex
% \section{Anhang}
% \subsection{Schritt-für-Schritt Anleitung}
% \label{app:Anleitung}
% gesetzt in der pagecommand-Option von \includepdf, damit die erste PDF-Seite die richtigen Header bekommt
\includepdf[scale=0.8,clip,trim=0cm 0cm 0cm 0cm,offset=0 -2cm,pages={1},pagecommand={\section{Anhang}\subsection{Schritt-für-Schritt Anleitung}\label{app:Anleitung}}]{AufbauEinerDMZ.pdf}
\includepdf[scale=0.85,clip,trim=0cm 0cm 0cm 0cm,offset=0 -0.5cm,pages={2-5},pagecommand={}]{AufbauEinerDMZ.pdf}
\clearpage

\subsection{Detaillierte Zeitplanung}
\label{app:Zeitplanung}
\tabelleAnhang{ZeitplanungKomplett}
\clearpage

\input{Anhang/AnhangLastenheft.tex}
\input{Anhang/AnhangPflichtenheft.tex}
\clearpage

\subsection{Netzpläne}
\label{app:Netzplan}
Der Netzplan unserer \acs{DMZ} in der Projektumgebung
\begin{figure}[htb]
\centering
\includegraphicsKeepAspectRatio{PLZNetzplanProjektumgebung.png}{1}
\caption{Netzplan der DMZ (Arbeitsgruppe 9)}
\end{figure}
Der Netzplan unserer \acs{DMZ} in der Testumgebung
\begin{figure}[htb]
    \centering
    \includegraphicsKeepAspectRatio{PLZNetzplanTestumgebung.png}{1}
    \caption{Netzplan der erweiterten DMZ in unserer Testumgebung}
\end{figure}
\clearpage

\subsection{Beschreibung des Testaufbaues}
\label{app:Test}
\subsection{title}
\subsubsection{firewall.sh (auf dem Outside-Router)}
\label{app:Firewall-Outside}
\lstinputlisting[language=sh]{Listings/outside/firewall.sh}

\subsubsection{firewall.sh (auf dem Inside-Router)}
\label{app:Firewall-Inside}
\lstinputlisting[language=sh]{Listings/inside/firewall.sh}

