% !TEX root = ../Projektdokumentation.tex
\section{Dokumentation}
\label{sec:Dokumentation}
% Wie wurde die Anwendung für die Benutzer/Administratoren/Entwickler dokumentiert (\zB Benutzerhandbuch, API-Dokumentation)?
Da unser Auftraggeber bereits früh im Projekt seinen Wunsch nach einer den IHK-Richtlinien für Projektdokumentationen entsprechenden und bevorzugt mit einem Programm wie \LaTeX{} erstellten Dokumentation der Umsetzung Ausdruck verlieh und wir noch an der Erstellung der Testumgebung zum Abschluss des Projekts arbeiteten, beschlossen wir uns, seinem Wunsch zu entsprechen. Im Nachhinein betrachtet hätte uns dieser völlig fehlkalkulierte Einsatz einer unbekannten Programmiersprache zur Umsetzung eines essentiellen Projektzieles beinahe das sprichwörtliche Genick gebrochen. Das Ergebnis mag sich zwar sehen lassen, dennoch schlägt die Bearbeitung der Dokumentation dank der aufgetretenen Schwierigkeiten im Umgang mit \LaTeX{} mit einem fast untragbar hohen Anteil des Zeitbudgets zu Buche. Nichtsdestotrotz ist dies hier das beschriebene Resultat. Wir hoffen nur, es war die Mühen wert.
% Hinweis: Je nach Zielgruppe gelten bestimmte Anforderungen für die Dokumentation (\zB keine IT-Fachbegriffe in einer Anwenderdokumentation verwenden, aber auf jeden Fall in einer Dokumentation für den IT-Bereich).

\paragraph*{Entwicklerdokumentation:}
Die der neben der Konfiguration angelegte Entwicklerdokumentation befindet sich im \Anhang{app:Anleitung}. Sie wurde als Schritt-für-Schritt-Anleitung zum Wiederherstellen des bereits erreichten Zustandes im Fall eines technischen Versagens geführt.
%Die Entwicklerdokumentation wurde mittels PHPDoc\footnote{Vgl. \cite{phpDoc}} automatisch generiert. Ein beispielhafter Auszug aus der Dokumentation einer Klasse findet sich im \Anhang{app:Doc}. 

\Zwischenstand{Dokumentation}{Dokumentation}
