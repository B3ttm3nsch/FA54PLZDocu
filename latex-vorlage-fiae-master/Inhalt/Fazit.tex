% !TEX root = ../Projektdokumentation.tex
\section{Fazit} 
\label{sec:Fazit}

\dots

\subsection{Soll-/Ist-Vergleich}
\label{sec:SollIstVergleich}
% Wurde das Projektziel erreicht und wenn nein, warum nicht?

\dots
%ja, viel über Netzwerk, Firewall(iptables), Linux(grundsätzliche Struktur, Terminal)
%aber nicht in der Zeit
%ne


% Ist der Auftraggeber mit dem Projektergebnis zufrieden und wenn nein, warum nicht?
% Wurde die Projektplanung (Zeit, Kosten, Personal, Sachmittel) eingehalten oder haben sich Abweichungen ergeben und wenn ja, warum?

\dots
% zeit, somit kosten
% auch aufgrund von Krankheit, begrenztem Zugang
% testabweichung

% Hinweis: Die Projektplanung muss nicht strikt eingehalten werden. Vielmehr sind Abweichungen sogar als normal anzusehen. Sie müssen nur vernünftig begründet werden (\zB durch Änderungen an den Anforderungen, unter-/überschätzter Aufwand).

%\paragraph{Beispiel (verkürzt)}
Wie in Tabelle~\ref{tab:Vergleich} zu erkennen ist, konnte die Zeitplanung bis auf wenige Ausnahmen, einige davon jedoch aus bereits unter \ref{sec:Dokumentation} erwähnten Gründen mit gravierend abweichenden Zeiten, eingehalten werden (falls der Auftraggeber bei unserer verspäteten Abgabe nochmal beide Augen zudrückt).

\tabelle{Soll-/Ist-Vergleich}{tab:Vergleich}{Zeitnachher.tex}


\subsection{Lessons Learned}
\label{sec:LessonsLearned}
% Was hat der Prüfling bei der Durchführung des Projekts gelernt (\zB Zeitplanung, Vorteile der eingesetzten Frameworks, Änderungen der Anforderungen)?

\dots
% Zeitplanung, Linux, Firewall, zusätzlich (Netzwerk-)Virtualisierung

\subsection{Ausblick}
\label{sec:Ausblick}
%  Wie wird sich das Projekt in Zukunft weiterentwickeln (\zB geplante Erweiterungen)?

\dots
%Netzwerk ausbauen, domain controller, DHCP, DNS, FTP, Exchange (über Windows und / oder Linux)
