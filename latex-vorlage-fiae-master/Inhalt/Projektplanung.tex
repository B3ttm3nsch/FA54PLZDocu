% !TEX root = ../Projektdokumentation.tex
\section{Projektplanung} 
\label{sec:Projektplanung}

Da unser Projekt über die Dauer eines ganzen Schuljahres angelegt ist und wir die Unterrichtszeit zum Teil mit dem Erlernen von Fertigkeiten im Umgang mit Linux verbringen werden, muss der Ablauf genau geplant werden. Im folgenden erläutern wir die einzelnen Projektphasen, welche Ressourcen genutzt wurden und wann die Durchführung von der Planung abgewichen ist.

\subsection{Projektphasen}
\label{sec:Projektphasen}

    % In welchem Zeitraum und unter welchen Rahmenbedingungen (\zB Tagesarbeitszeit) findet das Projekt statt?
    Im Rahmen des P/LZ Unterrichts erhalten wir in jeder Schulwoche meist Freitags für je zwei Blöcke a 90 Minuten Zugang zum Labor 3.1.01 am OSZ IMT in Berlin. Das Schuljahr umfasst 14 Schulwochen in denen das Projekt durchgeführt werden muss. Außerhalb der Schulzeit können wir Private Ressourcen nutzen und planen pro Schulwoche jeweils 6 Stunden Freizeit am Wochenende als zusätzliche Pufferzeit ein. Die 42 Laborstunden und die Pufferzeit von 84 Stunden ergeben eine Gesamtzeit von 126 Stunden bis zur Projektabgabe.\\
	% Verfeinerung der Zeitplanung, die bereits im Projektantrag vorgestellt wurde.
    Wir gehen davon aus die grundlegende Planung und Analyse in den ersten beiden Schulwochen durchzuführen, die nächsten drei Schulwochen sollte das Netzwerk entworfen und erstellt werden. Anschließend wollen wir mit der Implementierung der Firewall beginnen, wofür wir \ca vier Schulwochen einplanen. Die Restliche Schulzeit wird für die Erstellung der Dokumentation und eine Stunde für die Abnahme durch den Kunden verplant. Je nach Bedarf kann die Pufferzeit zu weiterer Recherche zuhause genutzt werden.

\subsection{Zeitplanung}
\label{sec:Zeitplanung}

Tabelle~\ref{tab:Zeitplanung} zeigt unsere Zeitplanung für die einzelnen Projektphasen:
\tabelle{Zeitplanung}{tab:Zeitplanung}{ZeitplanungKurz}

\subsection{Abweichungen vom Projektantrag}
\label{sec:AbweichungenProjektantrag}

	% Sollte es Abweichungen zum Projektantrag geben (\zB Zeitplanung, Inhalt des Projekts, neue Anforderungen), müssen diese explizit aufgeführt und begründet werden.
    Aufgrund unserer Unerfahrenheit im Umgang mit \LaTeX{} gestaltet sich die Erstellung der Projektdokumentation leider schwieriger als vermutet. Zudem konnten die Funktionstests an unserer Firewall nicht bis zum Ende des letzten Unterrichtsblockes abgeschlossen werden, worauf Herr Krüger viel Zeit damit verbracht hat, eine zweite Testumgebung für unser Firewall-Script mit Windows Server 2016 zu virtualisieren, deren Installation und Konfiguration im Anhang dokumentiert wurde. Deshalb erbaten wir  eine kurzzeitige Verlängerung der Abgabefrist und konnten nur die ebenfalls im Anhang zu findende Schritt-für-Schritt Anleitung einsenden, welche während des Unterrichtes erstellt und benutzt wurde.

\subsection{Ressourcenplanung}
\label{sec:Ressourcenplanung}

	% Detaillierte Planung der benötigten Ressourcen (Hard-/Software, Räumlichkeiten \usw).
Für die Durchführung im Labor werden benötigt:
\begin{itemize}
    \item 2 Rechner mit Windows (und einem Benutzeraccount mit Adminrechten)
    \item die Software VMWare Player
    \item ein Distribution von Debian für die virtuelle Maschine
    \item Zugang zum Labornetz
    \item Einen Webserver und einen Editor zum Bearbeiten von HTML
    \item Zugang zum Internet zur Recherche
    \item Software zum Festhalten von Ergebnissen
    \item Software zum Durchführen von Tests
   	% \Ggfs sind auch personelle Ressourcen einzuplanen (\zB unterstützende Mitarbeiter).
    \item Unterstützung durch fachkundige Mitschüler wie Herrn Habekost, Herrn Schernekau und Herrn Mahnke
    \item Hilfe durch Herrn Henze bei schwereren Problemen
\end{itemize}
    % Hinweis: Häufig werden hier Ressourcen vergessen, die als selbstverständlich angesehen werden (\zB PC, Büro). 
    Für die zusätzliche Recherche zuhause hat Herr Krüger (todo). Herr Biller kann seine Recherchen und weitere Versuche zuhause sowohl auf einem Rechner mit Ubuntu 14.04 als auch auf einem Rechner mit Windows 7 im eigenen Heimnetz mit Internetanbindung durchführen. Beide haben jeweils die benötigte Software um die im Labor anfallenden Aufgaben durchführen zu können sowie \LaTeX{} und benötigte Editoren, um die Dokumentation anzufertigen. Dank einer während des Projektes geführten Schritt-für-Schritt Anleitung zum Einrichten des Netzwerks (und der Möglichkeit, virtuelle Maschinen zu speichern und zu kopieren) kann am Projekt auch zuhause gearbeitet werden.

\subsection{Entwicklungsprozess}
\label{sec:Entwicklungsprozess}

	% Welcher Entwicklungsprozess wird bei der Bearbeitung des Projekts verfolgt (\zB Wasserfall, agiler Prozess)?
    Um unser Projekt zu planen und durchzuführen benutzen wir ein Kanban-Board, also eine Version von SCRUM.
