% !TEX root = ../Projektdokumentation.tex
\section{Entwurfsphase} 
\label{sec:Entwurfsphase}

% Erklärung
Da unsere Hard- und Software von unserem Auftraggeber gestellt und vorgegeben wird, erübrigt seine ausführliche Begründung, weshalb wir diese Materialien verwendet haben. Zudem wird so sichergestellt, dass während unserer Projektzeit alle benötigten Mittel zur Verfügung stehen.

\subsection{Zielplattform}
\label{sec:Zielplattform}

\paragraph*{Hardware: } 
Die uns zur Verfügung stehenden Desktop PCs bleiben unverändert. Die Leistungsdaten derer  genügen für den Aufbau einer einfachen DMZ.

\paragraph*{Software: } 
Für die Implementation eines Routers als virtuelle Maschine nutzen wir den vorinstallierten VMWare Player. Dieser ist kostenlos und berechtigt uns zum Virtualisieren einer Linux Distribution. Des Weiteren werden wir auch das beigefügte Debian benutzen. Auf den VMs wird mit BASH und Linux-Befehlen gearbeitet, da wir nur kleinere Konfigurationen und Scripts schreiben. Um die Konfiguration zu testen, die Router per Remote zu konfigurieren und eventuell Dateien auszutauschen, wird noch SSH- und FTP-Client-Software benötigt. Dafür werden wir Putty und winscp verwenden. Diese Tools sind kompakt und beeinträchtigen nicht die Leistung der Hosts.

\subsection{Netzwerkplan}
\label{sec:Geschaeftslogik}

\Anhang{app:Netzplan} zeigt die grundsätzliche IP-Adressverteilung in den geplanten Netzwerken.
Der zweite Netzplan zeigt die erweiterte Testumgebung. 
Unser Konzept teilt sich jeweils grundsätzlich in das Labornetz (hier symbolisch für den Rest der Welt), das interne Netz (mit den Windows-Clients unseres Kunden) und das von der Außenwelt abgeschottete \ac{DMZ}-Netzwerk, welches nur über spezielle Berechtigungen zu erreichen und für spezielle Dienste (Webserver) zu verwenden ist.

\subsection{Maßnahmen zur Qualitätssicherung}
\label{sec:Qualitaetssicherung}
%begin{itemize}
	%\item Welche Maßnahmen werden ergriffen, um die Qualität des Projektergebnisses (siehe Kapitel~\ref{sec:Qualitaetsanforderungen}: \nameref{sec:Qualitaetsanforderungen}) zu sichern (\zB automatische Tests, Anwendertests)?
%	\item \Ggfs Definition von Testfällen und deren Durchführung (durch Programme/Benutzer).
%\end{itemize}
Bei jeder Veränderungen der Konfiguration werden Funktionstests durchgeführt. Diese sollen gewährleisten, dass die Anforderungen aus dem \nameref{sec:Lastenheft} eingehalten werden. Vorgenommene Änderungen an der Firewall und der Systemkonfiguration wird in unserer \Anhang{app:Anleitung} notiert und das Firewall-Script wird auf einem externen Datenträger gespeichert. So wird sichergestellt, dass auch bei einem Defekt die ursprüngliche Konfiguration schnell wieder herstellbar ist.

\subsection{Pflichtenheft}
\label{sec:Pflichtenheft}
	%\item Auszüge aus dem Pflichtenheft/Datenverarbeitungskonzept, wenn es im Rahmen des Projekts erstellt wurde.
Die aus den Anforderungen des aus dem Lastenheft \Anhang{app:Lastenheft} zu findenden hervorgegangenen Anforderungen werden im \Anhang{app:Pflichtenheft} genauer erläutert.

%\paragraph{Beispiel}
%Ein Beispiel für das auf dem Lastenheft (siehe Kapitel~\ref{sec:Lastenheft}: \nameref{sec:Lastenheft}) aufbauende Pflichtenheft ist im \Anhang{app:Pflichtenheft} zu finden.


\Zwischenstand{Entwurfsphase}{Entwurf}
