% !TEX root = ../Projektdokumentation.tex
\section{Entwurfsphase} 
\label{sec:Entwurfsphase}

% Erklärung
Da unsere Hard- und Software von unserem Auftraggeber gestellt und vorgegeben wird, erübrigt seine ausführliche Begründung, weshalb wir diese Materialien verwendet haben. Zudem wird so sichergestellt, dass während unserer Projektzeit alle benötigten Mittel zur Verfügung stehen.

\subsection{Zielplattform}
\label{sec:Zielplattform}

\paragraph*{Hardware: } 
Die uns zur Verfügung stehenden Desktop PCs bleiben unverändert. Die Leistungsdaten derer  genügen für den Aufbau einer einfachen DMZ.

\paragraph*{Software: } 
Für die Implementation eines Routers als virtuelle Maschine nutzen wir den vorinstallierten VMWare Player. Dieser ist kostenlos und berechtigt uns zum Virtualisieren einer Linux Distribution. Des Weiteren werden wir auch das beigefügte Debian benutzen. Auf den VMs wird mit BASH und Linux-Befehlen gearbeitet, da wir nur kleinere Konfigurationen und Scripts schreiben. Um die Konfiguration zu testen, die Router per Remote zu konfigurieren und eventuell Dateien auszutauschen, wird noch SSH- und FTP-Client-Software benötigt. Dafür werden wir Putty und winscp verwenden. Diese Tools sind kompakt und beeinträchtigen nicht die Leistung der Hosts.

\subsection{Netzwerkplan}
\label{sec:Geschaeftslogik}

\Abbildung{Netzplan} zeigt die grundsätzliche IP-Adressverteilung in den geplanten Netzwerken. 
Unser Konzept teilt sich grundsätzlich in das Labornetz (hier symbolisch für den Rest der Welt), das interne Netz (mit den Windows-Clients unseres Kunden) und das von der Außenwelt abgeschottete \ac{DMZ}-Netzwerk, welches nur über spezielle Berechtigungen zu erreichen und für spezielle Dienste (Webserver) zu verwenden ist.
\begin{figure}[htb]
\centering
\includegraphicsKeepAspectRatio{PLZNetzplanProjektumgebung.png}{0.9}
\caption{Netzplan DMZ Arbeitsgruppe 9}
\label{fig:Netzplan}
\end{figure}


\subsection{Maßnahmen zur Qualitätssicherung}
\label{sec:Qualitaetssicherung}
%begin{itemize}
	%\item Welche Maßnahmen werden ergriffen, um die Qualität des Projektergebnisses (siehe Kapitel~\ref{sec:Qualitaetsanforderungen}: \nameref{sec:Qualitaetsanforderungen}) zu sichern (\zB automatische Tests, Anwendertests)?
%	\item \Ggfs Definition von Testfällen und deren Durchführung (durch Programme/Benutzer).
%\end{itemize}
Bei jeder Veränderungen der Konfiguration werden Tests durchgeführt. Diese sollen gewährleisten, dass das \nameref{sec:Lastenheft} eingehalten wird. Vorgenommene Konfigurationen werden notiert und das Firewall-Script wird zusätzlich auf einen externen Datenträger kopiert. So wird sichergestellt, dass bei einem Defekt die ursprüngliche Konfiguration schnell wieder verfügbar ist.

\subsection{Pflichtenheft/Datenverarbeitungskonzept}
\label{sec:Pflichtenheft}
	%\item Auszüge aus dem Pflichtenheft/Datenverarbeitungskonzept, wenn es im Rahmen des Projekts erstellt wurde.
\begin{enumerate}

	\item Musskriterien
	
	\begin{itemize}
		\item Das DMZ-Netz erhält die Netzmaske 172.16.9.0/24
		\item Das intere Netz erhält die Netzmaske 10.0.9.0/24
		\item Die öffentliche Schnittstelle des Outside-Router erhält die IP 192.168.200.109
		\item Der Outside-Router erhält als Standard-Gateway die IP 192.168.200.1
		\item Der Outside-Router erhält eine statische Route für das interne und DMZ-Netz
		\item Der Inside-Router erhält als Standard-Gateway das Interface des Outside-Routers, welches in die DMZ zeigt
		\item Der Webserver ist über die öffentliche IP des Outside-Routers über HTTP/S von außen erreichbar
		\item Der Webserver ist über die lokale IP 172.16.9.3 über HTTP/S aus dem internen Netzwerk erreichbar
		\item Die Router und Windows-Clients bekommen als DNS-Server die IPs 192.168.95.40 und 192.168.95.41
		\item Die Router und Windows-Clients bekommen als NTP-Server die IP 192.168.200.1
		\item Die Firewall verhindert unrechtmäßigen Datentransfer zwischen den Netzen und auf den Routern
		\item Der Admin-PC mit der IP 10.0.9.2 ist berechtigt mittels SSH auf die Router zuzugreifen	
	\end{itemize}
	
	\item Kannkriterien
	
	\begin{itemize}
		\item Die Firewall lässt sich mit den Optionen "start" und "stop" an- bzw.\ ausschalten
		\item Die Firewall-Scripts der Router befinden sich im Verzeichnis /root/bin
		\item Die Veränderung der Firewall-Konfiguration befindet sich jeweils im Verzeichnis /var/log/firewall
		\item Der Admin-PC mit der IP 10.0.9.2 ist berechtigt mittels RDP auf den Webserver zuzugreifen
	\end{itemize}
\end{enumerate}

%\paragraph{Beispiel}
%Ein Beispiel für das auf dem Lastenheft (siehe Kapitel~\ref{sec:Lastenheft}: \nameref{sec:Lastenheft}) aufbauende Pflichtenheft ist im \Anhang{app:Pflichtenheft} zu finden.


\Zwischenstand{Entwurfsphase}{Entwurf}
