% !TEX root = ../Projektdokumentation.tex
\section{Implementierungsphase} 
\label{sec:Implementierungsphase}
Bevor mit der eigentlichen Programmierung und Implementierung begonnen werden kann, muss direkt am Anfang eine dazugehörige Entwicklungsumgebung aufgesetzt werden. Dies war zum Zeitpunkt des Projektantrags nicht bekannt. 
Während der gesamten Dauer das Projekts wurde regelmäßig über die Versionsverwaltungssoftware Github gesichert.

\subsection{Implementierung der Projektumgebung}
\label{subsec:ImplementierungDatenstruktur}
% Beschreibung der angelegten Datenbank (z.B. Generierung von SQL aus Modellierungswerkzeug oder händisches Anlegen), XML-Schemas usw..
Nach einem Backup des bisherigen Systems auf einem MacBook Pro, werden Ruby, Rails und alle weiteren benötigten Frameworks und Bibliotheken mit den bekannten Informationen eingebunden und deren Funktionstüchtigkeit getestet. Der MySQL-Server, Ruby Version Manager (rvm), über den Ruby installiert wird, sowie das Datenbank-Management System Sequel Pro werden mit Hilfe des OS X Paket Managers Homebrew installiert. Über den Bash rails new app wird ein neues Projekt erstellt. Als Entwicklungsumbegung (IDE) wird Rubymine verwendet.


\subsection{Implementierung des Webservers und des Routings}
\label{subsec:ImplementierungBenutzeroberfläche}
% Beschreibung der Implementierung der Benutzeroberfläche, falls dies separat zur Implementierung der Geschäftslogik erfolgt (z.B. bei HTML-Oberflächen und Stylesheets).
% Ggfs. Beschreibung des Corporate Designs und dessen Umsetzung in der Anwendung.
% Screenshots der Anwendung
Als Web Server wird Thin verwendet, welcher leicht als RubyGem über Ruby installiert werden kann und keiner weiteren Konfiguration bedarf. In der Projektdatei config/routes.rb werden die Routen zum Controller mit den entsprechenden Operationen definiert.

\subsection{Implementierung der Datenbank}
\label{subsec:ImplementierungGeschäftslogik}
% Beschreibung des Vorgehens bei der Umsetzung/Programmierung der entworfenen Anwendung. 
% Ggfs. interessante Funktionen/Algorithmen im Detail vorstellen, verwendete Entwurfsmuster zeigen. 
% Quelltextbeispiele zeigen. 
% Hinweis: Wie in Kapitel 1: Einleitung zitiert, wird nicht ein lauffähiges Programm bewertet, sondern die Projektdurchführung. Dennoch würde ich immer Quelltextausschnitte zeigen, da sonst Zweifel an der tatsächlichen Leistung des Prüflings aufkommen können.
% Beispiel Die Klasse ComparedNaturalModuleInformation findet sich im Anhang A.11: Klasse: ComparedNaturalModuleInformation auf Seite xv.
Nach der Erstellung eines Projekts wird über den simplen Befehl rake db:create 
die Datenbank erstellt werden und danach das vom externen Entwickler erhaltene MySQL-Skript importiert. Migrationen für die neuen Tabellen werden in Rails geschrieben und über rake db:migrate der Datenbank hinzugefügt. Das hat den Vorteil, dass man jederzeit seine selbst erstellten Migrationen zurücksetzen, diese editieren und der Datenbank wieder hinzufügen kann. Über ein erstelltes MySQL-Skript werden dann die Daten für die Tabellen der Dropdown Menüs importiert.

\subsection{Implementierung der Models}
Für jede Tabelle der Datenbank wird eine Klasse erstellt. Dies geschieht über den Befehl rails generate model [Tabellenname]. Dabei ist darauf zu achten, dass man sich an den Namenskonventionen von Rails orientiert. Rails kann so die entsprechenden Modelle der jeweiligen Tabelle automatisch zuordnen und die Datenbankabfragen generieren. Danach werden in den Model-Klassen die Relationen der Klassen nach dem Vorbild der Tabellen untereinander definiert und Operationen zur Gültigkeitsprüfung der an die Tabelle zu übergebenen Daten geschrieben.

\subsection{Implementierung des Fachkonzepts}
Über den Befehl rails generate controller [Controllername] wird eine neue Klasse für die entsprechenden Controller erstellt. Entsprechend der Namenskonventionen werden durch Rails zusätzliche Hilfsdateien, sowie einen Pfad zur View des Controllers angelegt. Anhand der Methodennamen erfolgt das Routing zu dessen Operationen. Des Weiteren wird anhand der Rails-Namenskonvention auch die entsprechende Ansicht festgelegt. Hier werden die entsprechenden CRUD – Operationen implementiert. 

\subsection{Implementierung der Benutzeroberfläche}
Für die Ansicht der Projekte und Teilprojekte werden nur die nötigsten Dateien und Links für die Weiterleitung zum Materialeingangsbericht implementiert. Die Ansichten für den Bericht werden mit der HTML – Abstraktionssprache HAML geschrieben. Über die Index-Datei des Berichts werden alle zu dem Teilprojekt aufgelisteten gespeicherten Berichte angezeigt und zu den entsprechenden Ansichten (Anzeigen, Bearbeiten, Löschen) verlinkt. Zudem wird ein Link zur Erstellung eines neuen Berichts generiert. Für das Anzeigen, Bearbeiten und Erstellen wird jeweils das gleiche „Partial“ – Formular mit einem Parameter für das Aktivieren bzw. Deaktivieren geladen. Die Datenauswahl in den Dropdown-Menüs werden über Rails geladen.

\Zwischenstand{Implementierungsphase}{Implementierung}
