% !TEX root = ../Projektdokumentation.tex
\section{Analysephase} 
\label{sec:Analysephase}

% Überblick
Im Nachfolgenden verzichten wir auf einen Großteil der üblichen Berechnungen zur Wirtschaftlichkeit des Projektes, da dieses zum Großteil unserer fachlichen Kompetenzbildung dienen soll. Darüber hinaus wäre für ein fiktives mittelständisches Unternehmen ein bereits existierendes Produkt sowohl vom zu erwartenden Arbeitsaufwand wie auch finanziell deutlich günstiger. Es wird daher lediglich eine Beispielhafte Kostenberechnung für die Umsetzung der Planung durch uns erstellt und dafür ein größeres Augenmerk auf Anforderungen und Nutzen des Projekts gelegt. 

\subsection{Ist-Analyse} 
\label{sec:IstAnalyse}

	% Wie ist die bisherige Situation (\zB bestehende Programme, Wünsche der Mitarbeiter)?
    \textbf{Was ist vorhanden:}\\
    Im Labor sind für jedes Gruppenmitglied vorhanden: ein Bildschirmarbeitzplatz, Windows 7, Adminrechte, zwei physikalische Netzwerkinterfaces, Anschluß an Labornetzwerk und Internet, die Software VMWare Player, Debian Images auf einem Netzlaufwerk.
    
	% Was gilt es zu erstellen/verbessern?
    \textbf{Was ist zu erstellen:}\\
    Zuerst muss nun von jeder Gruppe ein Netzplan erstellt werden. Dann gilt es, die Debian 7 (Wheezy) Linux-Images in virtuellen Maschinen auf beiden Rechnern mit Hilfe des VMWare Players aufzusetzen. Diese werden zu einem Outside- und einem Inside-Router konfiguriert und die geplanten Netzwerk- und Routingeinstellungen müssen sowohl an den virtuellen wie auch physikalischen Schnittstellen durchgeführt werden. Auf dem Rechner des Outside-Routers muss ein Webserver eingerichtet werden, wofür NAT und Port-Forwarding nötig sind. Zwischendurch wird es immer wieder der gezielten Recherche bedürfen. Um schließlich Zugriffe von außen zu regulieren, muss eine Firewall mit entsprechenden Regeln erstellt wwerden, die per Skript an- und abschaltbar ist. Die Funktionalität muss getestet werden und Projekt und Tests sind zu dokumentieren. Unser Lernfortschritt ist in einem Kompetenzportfolio niederzuschreiben. Gleichzeitig sind Laborübungen und Tests zu Linux-Kenntnissen zu absolvieren.

\subsection{Wirtschaftlichkeitsanalyse}
\label{sec:Wirtschaftlichkeitsanalyse}
    Wie bereits Anfänglich erwähnt, lohnt sich das Projekt für ein fiktives mittelständisches Unternehmen nur bedingt.

\subsubsection{\gqq{Make or Buy}-Entscheidung}
\label{sec:MakeOrBuyEntscheidung}
	% Gibt es vielleicht schon ein fertiges Produkt, dass alle Anforderungen des Projekts abdeckt?
    Die Kosten für eine qualifizierte Kraft zur ständigen Wartung des Servers, die durch Dauerbetrieb anfallenden Stromkosten sowie die zusätzlichen Hardwarekosten bei einem zukünftigen Upscaling übersteigen bei weitem die Kosten für einen fachkundig und sicher Administrierten Server bei einem seriösen Hosting-Anbieter.\\
	% Wenn ja, wieso wird das Projekt trotzdem umgesetzt?
    Da unsere Empfehlung an den Kunden ein Produkt eines anderen Anbieters wäre, wird das Projekt nur zu unserem Nutzen und der Erfahrung willen, die wir damit gewinnen, umgesetzt.

\subsubsection{Projektkosten}
\label{sec:Projektkosten}
	% Welche Kosten fallen bei der Umsetzung des Projekts im Detail an (\zB Entwicklung, Einführung/Schulung, Wartung)?
    Da es sich nur um ein fiktives Projekt handelt, verzichten wir auf eine detaillierte Berechnung mit Stromkosten innerhalb des Labors, den Gehältern der Lehrkräfte oder etwaiger Lizenzgebühren. Wir beschränken uns auf eine fiktive Beispielrechnung mit unserem Stundenlohn während der Projektdauer.

\paragraph{Beispielrechnung (verkürzt)}
Die realen Kosten für die Durchführung des Projekts setzen sich sowohl aus Personal-, als auch aus Ressourcenkosten zusammen. Wir rechnen hier lediglich mit dem fiktiven Gehalt eines Auszubildendem im zweiten Lehrjahr von ca. \eur{800} Brutto pro Monat. 

% Stundenlohn (brutto) = 3 × dein Monatslohn (brutto) ÷ 13 ÷ die Anzahl deiner wöchentlichen Arbeitsstunden
\begin{eqnarray}
3\cdot \eur{800}\mbox{/Monat} \div {13} \div 40 \mbox{ h/Monat} \approx \eur{4,62}\mbox{/h}
\end{eqnarray}

Es ergibt sich also ein Stundenlohn von \eur{{4,62}. 
Die Durchführungszeit des Projekts beträgt 42 Stunden. 
Die Nutzung von Ressourcen\footnote{Räumlichkeiten, Arbeitsplatzrechner etc.} sowie die Kosten durch andere Mitarbeiter werden hier nicht mit eingerechnet. 
Eine Aufstellung der Kosten befindet sich in Tabelle~\ref{tab:Kostenaufstellung} und sie betragen insgesamt \eur{388,08}.
\tabelle{Kostenaufstellung}{tab:Kostenaufstellung}{Kostenaufstellung.tex}


\subsubsection{Amortisationsdauer}
\label{sec:Amortisationsdauer}

	% Welche monetären Vorteile bietet das Projekt (\zB Einsparung von Lizenzkosten, Arbeitszeitersparnis, bessere Usability, Korrektheit)?
	% Wann hat sich das Projekt amortisiert?
    Aufgrund unserer \gqq{Make or Buy}-Entscheidung und da das Projekt nur zu Lernzwecken umgesetzt wird verzichten wir hier auf die Berechnung eines fiktiven Rentabilitätszeitpunktes. Das gelernte wird sich spätestens zur IHK-Prüfung und bei der Anfertigung der Dokumentation des IHK-Abschlussprojektes auszahlen.

\subsection{Nutzwertanalyse}
\label{sec:Nutzwertanalyse}
	% Darstellung des nicht-monetären Nutzens (\zB Vorher-/Nachher-Vergleich anhand eines Wirtschaftlichkeitskoeffizienten). 
    \dots

\paragraph{Beispiel}
Ein Beispiel für eine Entscheidungsmatrix findet sich in Kapitel~\ref{sec:Architekturdesign}: \nameref{sec:Architekturdesign}.


\subsection{Anwendungsfälle}
\label{sec:Anwendungsfaelle}
	% Welche Anwendungsfälle soll das Projekt abdecken?
    \dots
	% Einer oder mehrere interessante (!) Anwendungsfälle könnten exemplarisch durch ein Aktivitätsdiagramm oder eine \ac{EPK} detailliert beschrieben werden. 
    \dots

\paragraph{Beispiel}
Ein Beispiel für ein Use Case-Diagramm findet sich im \Anhang{app:UseCase}.


\subsection{Qualitätsanforderungen}
\label{sec:Qualitaetsanforderungen}
	% Welche Qualitätsanforderungen werden an die Anwendung gestellt (\zB hinsichtlich Performance, Usability, Effizienz \etc (siehe \citet{ISO9126}))?
    \dots

\subsection{Lastenheft/Fachkonzept}
\label{sec:Lastenheft}
	% Auszüge aus dem Lastenheft/Fachkonzept, wenn es im Rahmen des Projekts erstellt wurde.
    \dots
	% Mögliche Inhalte: Funktionen des Programms (Muss/Soll/Wunsch), User Stories, Benutzerrollen
    \dots

\paragraph{Beispiel}
Ein Beispiel für ein Lastenheft findet sich im \Anhang{app:Lastenheft}. 

\Zwischenstand{Analysephase}{Analyse}
